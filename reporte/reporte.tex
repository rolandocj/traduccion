\documentclass[12pt, letterpaper]{article}
\usepackage[colorlinks=true,linkcolor=black,citecolor=blue,filecolor=cyan,pagecolor=blue]{hyperref} 
\usepackage[toc,style=altlistgroup,hyperfirst=false]{glossaries}
\usepackage[utf8]{inputenc} %para poder escribir símbolos no anglosajones 
\usepackage[spanish, mexico]{babel} %Escribir en español (acentos)
\usepackage[T1]{fontenc}
\usepackage{amssymb}
\usepackage{mathtools}
\usepackage[usenames]{color}
\usepackage{float}
\usepackage{graphicx}  %%para las imagenes
\usepackage{cite} % para contraer referencias
\usepackage{multicol}
\usepackage{multirow}
\usepackage{bm}
\usepackage{bbm}
\usepackage[left=2.5cm,top=2.5cm,right=2.5cm,bottom=2.5cm]{geometry}
\parindent=5mm
\graphicspath{{images/}}
\usepackage{etoolbox}
\let\bbordermatrix\bordermatrix
%\patchcmd{\bbordermatrix}{8.75}{4.75}{}{}
%\patchcmd{\bbordermatrix}{\left(}{\left[}{}{}
%\patchcmd{\bbordermatrix}{\right)}{\right]}{}{}
%%%%glosario
\makeindex
%\makeglossaries
%\input{./glosario.tex}

%%%%%%%%%%%%%%%%%%%%%%%%%%%%%%%%%%%%%%%%%%%%%%%%%%%%%%%%%%%%%%%%%%%%%%%%%%%%%
%%NOTA IMPORTANTE:
%%Para relacionar el glosario.tex con este archivo
%%Es necesario abrir la terminal (Simbolo del sistema en windows)
%%Ir a la carpeta contenedora y escribir el siguiente comando:
%%makeindex -s PROYECTO_final.ist -t PROYECTO_final.glg -o PROYECTO_final.gls PROYECTO_final.glo
%%%%%%%%%%%%%%%%%%%%%%%%%%%%%%%%%%%%%%%%%%%%%%%%%%%%%%%%%%%%%%%%%%%%%%%%%%%%%

%%%% inicio del documento
\begin{document}

\thispagestyle{empty}

%%%%%%% portada

\thispagestyle{empty}

\begin{minipage}[c][0.1\textheight][c]{0.2\textwidth}
\begin{center}
    \includegraphics[width=4cm, height=4cm]{cimat}
\end{center}
\end{minipage}
\begin{minipage}[c][0.1\textheight][t]{0.8\textwidth}
\begin{center}
    {\hspace{2cm}\scshape Centro de Investigación en Matemáticas}
    \vspace{-.5cm}
\end{center}
\hspace*{1.0cm} \rule[0mm]{0.9\textwidth}{0.8mm}
\hspace*{1.17cm}   \rule[4mm]{0.9\textwidth}{0.1mm}
    \vspace{-1cm}
\begin{center}
    { \hspace{2cm}\scshape  Unidad Monterrey}
\end{center}
\end{minipage}

\begin{minipage}[c][0.6\textheight][t]{0.2\textwidth}
\begin{center}
\hskip2pt
\vrule width2.5pt height10cm
        \hskip1mm
        \vrule width1pt height10cm \\ \vspace{2cm}
        \includegraphics[height=4.5cm]{mty}
        \end{center}
\end{minipage}
\begin{minipage}[c][0.9\textheight][t]{0.65\textwidth}
  \begin{center}

	
    \vspace{3.2cm}
    
%%%% TITULO EN PORTADA

  \scshape Proyecto final.\\ \normalsize
  
  \vspace{2cm}  
  
    
            
    Temas selectos de ciencia de datos\\
    \vspace{1cm}   
    Machine Translation para lenguajes con pocos recursos.\\
    \vspace{1cm}   
    \vspace{1cm}   
    Ricardo Cruz Sánchez\\
    Rolando Corona Jiménez
    \vspace{.5cm}   
  \end{center}
  
\end{minipage}

%TABLA DE INDICES
\pagebreak
\tableofcontents

\cleardoublepage
%INTRODUCCIÓN
\pagebreak
\section{Introducción.}


\section{Machine Translation.}

\subsection{Modelo encoder-decoder}

\begin{figure}[h]
\centering
\includegraphics[scale=.5]{images/Encoder-Decoder-Architecture-for-Neural-Machine-Translation.png} 
\label{e-d}
\caption{Modelo encoder-decoder attention.}
\end{figure}

\section{Descripción del conjunto de datos.}



\section{Conclusiones.}


\begin{thebibliography}{1}

\bibitem{cr98}
Lorrie Faith Cranor and Brian A. LaMacchia. 1998. Spam!. Commun. ACM 41, 8 (August 1998), 74-83. 

\bibitem{fe19}
Emilio Ferrara. 2019. The history of digital spam. Commun. ACM 62, 8 (July 2019), 82-91. 

\bibitem{ha01}
Hastie, T., Tibshirani, R.,, Friedman, J. (2001). The Elements of Statistical Learning. New York, NY, USA: Springer New York Inc.. 

\bibitem{so09}
Marina Sokolova and Guy Lapalme. 2009. A systematic analysis of performance measures for classification tasks. Inf. Process. Manage. 45, 4 (July 2009), 427-437. 

\end{thebibliography}

\end{document}